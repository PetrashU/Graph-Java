\documentclass[]{article}
\usepackage{float}

\usepackage{lmodern}
\usepackage{graphicx}
\usepackage{amssymb,amsmath}
\usepackage{ifxetex,ifluatex}
\usepackage{fixltx2e} % provides \textsubscript
\ifnum 0\ifxetex 1\fi\ifluatex 1\fi=0 % if pdftex
  \usepackage[T1]{fontenc}
  \usepackage[utf8]{inputenc}
\else % if luatex or xelatex
  \ifxetex
    \usepackage{mathspec}
  \else
    \usepackage{fontspec}
  \fi
  \defaultfontfeatures{Ligatures=TeX,Scale=MatchLowercase}
\fi
\renewcommand{\figurename}{Rys.}
% use upquote if available, for straight quotes in verbatim environments
\IfFileExists{upquote.sty}{\usepackage{upquote}}{}
% use microtype if available
\IfFileExists{microtype.sty}{%
\usepackage[]{microtype}
\UseMicrotypeSet[protrusion]{basicmath} % disable protrusion for tt fonts
}{}
\PassOptionsToPackage{hyphens}{url} % url is loaded by hyperref
\usepackage[unicode=true]{hyperref}
\hypersetup{
            pdftitle={Generacja grafu i odnalezienie najkrótszej ścieżki pomiędzy węzłami: graph},
            pdfauthor={Ulyana Petrashevich, Inga Maziarz}
            pdfborder={0 0 0},
            breaklinks=true}
\urlstyle{same}  % don't use monospace font for urls
\IfFileExists{parskip.sty}{%
\usepackage{parskip}
}{% else
\setlength{\parindent}{0pt}
\setlength{\parskip}{6pt plus 2pt minus 1pt}
}
\setlength{\emergencystretch}{3em}  % prevent overfull lines
\providecommand{\tightlist}{%
  \setlength{\itemsep}{0pt}\setlength{\parskip}{0pt}}
\setcounter{secnumdepth}{0}
% Redefines (sub)paragraphs to behave more like sections
\ifx\paragraph\undefined\else
\let\oldparagraph\paragraph
\renewcommand{\paragraph}[1]{\oldparagraph{#1}\mbox{}}
\fi
\ifx\subparagraph\undefined\else
\let\oldsubparagraph\subparagraph
\renewcommand{\subparagraph}[1]{\oldsubparagraph{#1}\mbox{}}
\fi
\renewcommand{\contentsname}{Spis treści}
% set default figure placement to htbp
\makeatletter
\def\fps@figure{htbp}
\makeatother


\title{\texttt{Sprawozdanie końcowe}\\Generacja grafu i odnalezienie najkrótszej ścieżki pomiędzy węzłami - \texttt{graph} - Java}
\author{Ulyana Petrashevich, Inga Maziarz}
\date{08.06.2022}

\usepackage{fancyhdr}
\usepackage{lastpage}
\pagestyle{fancy}
\fancypagestyle{plain}{}
\fancyhf{}
\rhead{Sprawozdanie końcowe}
\lhead{Graph - Java}
\cfoot{Strona \thepage \hspace{1pt} z \pageref{LastPage}}

\begin{document}
\maketitle
\tableofcontents
\thispagestyle{empty}
\newpage

\section{Opis teoretyczny zagadnienia}\label{header-n231}

Program \texttt{Kratka} służy do wyszukiwania najkrótszej ścieżki w grafie ważonym nieskierowanym. Program zawiera funkcje wczytującą i wypisującą graf z/do pliku oraz generator grafu spójnego i niespójnego. Za wyszukiwanie najkrótszej ścieżki odpowiedzialne są funkcje \texttt{bfs} oraz \texttt{dijkstra}. \texttt{BFS}, czyli algorytm przeszukiwania wszerz, działa następująco: 
\begin{itemize}
\item
  Węzeł początkowy oznaczany jest jako odwiedzony. Do kolejki dodawane są sąsiadujące z nim węzły (w kolejności od węzła z najmniejszym indeksem).
\item
  Odwiedzony zostaje następny węzeł w kolejce. Postępowanie jest analogiczne do wcześniejszego; do kolejki zostają dodane węzły sąsiadujące z obecnym, jednak tylko te, które nie zostały odwiedzone wcześniej.
  
\item
  Proces jest powtarzany aż do momentu odwiedzenia wszystkich węzłów z kolejki. Jeżeli na końcu wszystkie węzły w grafie zostały odwiedzone, oznacza to, że przeszukiwany graf jest spójny. 
\end{itemize}
Jeżeli \texttt{bfs} zwróci wartość oznaczającą spójność grafu, to program przechodzi do kolejnego kroku, którym jest znalezienie najkrótszej ścieżki algorytmem Dijkstry, który działa w poniższy sposób:
\begin{itemize}
\item
Długości przy węźle początkowym otrzymują wartość 0. Długość ścieżki do każdego innego węzła zostaje ustawiona na nieskończoność. 
\item
  Oznaczamy węzeł początkowy jako odwiedzony. Dla każdego jego sąsiada zostaje przypisana długość równa wadze krawędzi między nimi. 
\item
  Nieodwiedzony węzeł o najmniejszej przypisanej długości zostaje oznaczony jako odwiedzony. Dla każdego jego sąsiada zostaje obliczona wartość równa sumie długości przy obecnym węźle i wagi krawędzi między nimi. Jeżeli znaleziona wartość jest mniejsza niż przypisana do sąsiada, to zostaje ona zamieniona. 
\item
  Poprzedni krok jest powtarzany aż do odwiedzenia wszystkich węzłów. Ostatecznie każdy węzeł (w tym wybrany jako końcowy) ma przypisaną długość najkrótszej ścieżki od węzła początkowego. Zapamiętana ścieżka może zostać wypisana na ekran.
\end{itemize}

\section{Opis wywołania}\label{header-n233}

Interakcja pomiędzy użytkownikiem a program odbywa się za pomocą interfejsu graficznego.

\begin{figure}[H]
\begin{center}
  \includegraphics[scale=0.5]{scren_GUI.png}
  \end{center}
  \caption{Wygląd głównego okna interfejsu}
  \label{fig:graf}
\end{figure}

\begin{figure}[H]
\begin{center}
  \includegraphics[scale=0.5]{scren_GUI_withgraf.png}
  \end{center}
  \caption{Wygląd głównego okna interfejsu po narysowaniu grafu}
  \label{fig:graf}
\end{figure}

\begin{figure}[H]
\begin{center}
  \includegraphics[scale=0.5]{scren_GUI_withpath.png}
  \end{center}
  \caption{Wygląd głównego okna interfejsu po narysowaniu ścieżki}
  \label{fig:graf}
\end{figure}

W górnym panelu znajdują się pola tekstowe, pozwalające wprowadzić pożądane cechy generowanego grafu lub będą wyświetlane parametry wczytanego grafu, w tym rozmiar, zakres wag krawędzi i spójność i przyciski, pozwalające uruchomić odpowiedni algorytm:
\begin{itemize}
    \item Generate
    
    Przycisk wczyta podane przez użytkownika w polach testowych i polu wyboru dane, sprawdzi je i wygeneruje odpowiedni graf. W przypadku, gdy poprzednio już został wygenerowany/wczytany graf lub wyliczona ścieżka, poprzednie dane zostaną usunięte.
    
    \item Read graph
    
    Przycisk uruchomi okno wyboru pliku. Postać pliku została opisana w podrozdziale \textbf{"Widok pliku"}. W przypadku udanego wczytania grafu program narysuje graf i ustawi odpowiednie wartości do pól tekstowych, etykiet i polu wyboru. W przypadku, gdy poprzednio już został wygenerowany/wczytany graf lub wyliczona ścieżka, poprzednie dane zostaną usunięte.
    
    \item Save graph
    
    Przycisk uruchomi okno wyboru ścieżki do utworzenia pliku i zapisanie do niego grafu w postaci tekstowej. Nie spowoduje usunięcie żadnych danych.
    \item Save paths
    
    Przycisk uruchomi okno wyboru ścieżki do utworzenia pliku i zapisanie do niego wszystkich wyznaczonych na rysunku ścieżek w postaci tekstowej. Nie spowoduje usunięcie żadnych danych.
    \item Clean paths
    
    Przycisk uruchomi wyczyszczenie wszystkich wyliczonych ścieżek i kosztów dojścia. Lecz nie spowoduje usunięcie grafu.
    \item Delete
    
    Przycisk spowoduje usunięcie wszystkich poprzednio wyliczonych danych, w tym grafu, kosztów i ścieżek.
\end{itemize}


Większą część okna zajmuje pole do wyświetlania grafu w postaci graficznej. 

\textbf{Najechanie myszką} na wierzchołek spowoduje pokazanie jego numeru. \textbf{Kliknięcie lewym klawiszem} myszki w okolicy wybranego wierzchołka spowoduje uruchomienie się algorytmu Dijkstry i zaznaczenie odległości do każdego innego wierzchołka. \textbf{Kliknięcie prawym klawiszem myszki} w okolicy innego wierzchołka spowoduje wyliczenie i narysowanie najkrótszej ścieżki do tego wierzchołka od wierzchołka, poprzednio wybranego za pomocą lewego klawisza.

Program pozwala na wyliczenie ścieżek od różnych wierzchołków. Dla tego po wyliczeniu ścieżki od jednego wierzchołka do drugiego, należy znowu wybrać wierzchołek początkowy za pomocą lewego klawisza i wierzchołek końcowy za pomocą prawego klawisza. Obie ścieżki będą się pokazywały i można będzie zapisać je do pliku.

Dolną część okienka zajmuje mapa kolorów z odpowiadającymi przedziałami wag krawędzi i kosztów dojścia. Przedziały będą się dynamicznie zmieniać wraz ze zmianą danych.
Pod mapą znajduje się przycisk "Modify color scale", uruchamiający okienko wyboru kolorów wartości minimalnych i maksymalnych dla mapy, koloru wierzchołków i koloru ścieżki, i opcja "Display nodes indices", uruchamiająca pokazanie numerów wierzchołków wewnątrz wierzchołków. 

\begin{figure}[H]
\begin{center}
  \includegraphics[scale=0.5]{scren_colors.png}
  \end{center}
  \caption{Wygląd okienka wyboru kolorów}
  \label{fig:graf}
\end{figure}

\subsection{Widok pliku}

//tu nie wiem, czy jest ważne aby było 2 spacje między i t.p., więc przeczytaj i coś popraw, jak coś

Program przyjmuje plik w formacie .txt.

Schemat poprawnego formatu danych:

Pierwszą linijkę zajmują dwie liczby: liczba wierszy i liczba kolumn, oddzielone od siebie minimum jedną spacją.

Zaczynając od drugiej linijki, podawane są krawędzie i ich wagi. Numer wierzchołka, od którego zaczyna się krawędź jest równy (numerowi linijki - 2), np. w 2 linijce, tuż po rozmiarach grafu, będą opisane krawędzie prowadzące do wierzchołka 0. Pierwsze dane o krawędzi muszą być podane, odstąpiwszy od początku linijki minimum jedną tabulację lub spację. Między numerem drugiego wierzchołka, złączonego krawędzią, a jej wagą musi być napisany chociażby jeden znak niepusty, np (:). Separatorem dziesiętnym wagi musi być kropka(.). Między dwoma krawędziami muszą być dwie lub więcej spacji. Tuż po ostatniej krawędzi, odnoszącej się do danego wierzchołka, przechodzimy na nową linijkę.

Nie ma potrzeby podawać wagi dwa razy symetrycznie, np. od 0 do 1 i od 1 do 0 – program duplikuje jedną z podanych wag przy wczytaniu.

Jeżeli do wierzchołka nie chcemy przypisywać żadnych krawędzi, pozostawiamy odpowiednią linijkę pustą – bez tabulacji, spacji lub innych znaków.

Przykład poprawnej formy danych:
\begin{verbatim}
2 2
    2 :2.76543  1 :4.56134

    1 :4.567345
    2 :6.47586
\end{verbatim}
W powyższym przykładzie wierzchołek numer 1 nie ma przypisanych krawędzi (choć będzie je miał po zduplikowaniu krawędzi od wierzchołków 0 i 2).
Postać symboliczna:

2\texttt{\verbvisiblespace}2\textbackslash n

\textbackslash t2\texttt{\verbvisiblespace}:2.76543\texttt{\verbvisiblespace}\texttt{\verbvisiblespace}1\texttt{\verbvisiblespace}:4.56134\textbackslash n

\textbackslash n

\textbackslash t1\texttt{\verbvisiblespace}:4.567345\textbackslash n

\textbackslash t2\texttt{\verbvisiblespace}:6.47586EOF

\section{Testy}\label{header-n233}
Podczas pisania programu były używane testy jednostkowe.
\begin{verbatim}
    

    @Test
    public void ConnectedGraphTrue1(){


        HashMap<Integer, ArrayList<Edge>> edges = new HashMap<>();
        ArrayList<Edge> tmp = new ArrayList<>();

        Edge edge1 = new Edge(1,1);
        Edge edge2 = new Edge(3,4);
        tmp.add(edge1);
        tmp.add(edge2);
        edges.put(0, (ArrayList<Edge>) tmp.clone());
        tmp.clear();

        edge1 = new Edge(0,1);
        edge2 = new Edge(2,3);
        tmp.add(edge1);
        tmp.add(edge2);
        edges.put(1,(ArrayList<Edge>) tmp.clone());
        tmp.clear();

        edge1 = new Edge(1,3);
        edge2 = new Edge(5,7);
        tmp.add(edge1);
        tmp.add(edge2);
        edges.put(2,(ArrayList<Edge>) tmp.clone());
        tmp.clear();

        edge1 = new Edge(0,4);
        tmp.add(edge1);
        edges.put(3,(ArrayList<Edge>) tmp.clone());
        tmp.clear();

        edge1 = new Edge(5,2);
        tmp.add(edge1);
        edges.put(4,(ArrayList<Edge>) tmp.clone());
        tmp.clear();

        edge1 = new Edge(4,2);
        edge2 = new Edge(2,7);
        tmp.add(edge1);
        tmp.add(edge2);
        edges.put(5,(ArrayList<Edge>) tmp.clone());
        tmp.clear();

        Graph graph = new Graph(2, 3, edges);

        assertTrue(graph.bfs());
    }
\end{verbatim}
\begin{verbatim}
    @Test
    void CheckGenerationOfConnectedGraph() {
        Graph graph = new Graph(3, 2);
        graph.generateGraph(true);
        assertTrue(graph.bfs());
    }
\end{verbatim}
\begin{verbatim}
    @Test
    void CheckGenerationOfNotConnectedGraph() {
        Graph graph = new Graph(3, 2);
        graph.generateGraph(false);
        assertFalse(graph.bfs());
    }
\end{verbatim}
\begin{verbatim}
    @Test
    void CheckReadingAndSaving(){
        Graph graph = new Graph(3,2);
        graph.generateGraph(true);
        File write;
        write = new File("filename.txt");
        try {
            PrintWriter w = new PrintWriter (write);
            graph.saveGraph(w);
            w.close();
        } catch (FileNotFoundException e) {
            throw new RuntimeException(e);
        }
        Graph graph1 = new Graph();
        Reader read;
        try {
            read = new FileReader(write);
            BufferedReader r = new BufferedReader(read);
            graph1.readGraph(r);
            r.close();
        } catch (IOException e) {
            throw new RuntimeException(e);
        }
        assertEquals(graph.row, graph1.row);
        assertEquals(graph.col, graph1.col);
        for (int i = 0; i < graph.edges.size(); i++)
            for (int j = 0; j < graph.edges.get(i).size(); j++) {
                assertEquals(graph.edges.get(i).get(j).fin, 
                                        graph1.edges.get(i).get(j).fin);
                assertEquals(graph.edges.get(i).get(j).weight, 
                                        graph1.edges.get(i).get(j).weight);

            }
    }
\end{verbatim}
\begin{verbatim}
    @Test
    void checkDijkstra() {
        HashMap<Integer, ArrayList<Edge>> edges = new HashMap<>();
        ArrayList<Edge> tmp = new ArrayList<>();

        Edge edge1 = new Edge(1,1);
        Edge edge2 = new Edge(3,4);
        tmp.add(edge1);
        tmp.add(edge2);
        edges.put(0, (ArrayList<Edge>) tmp.clone());
        tmp.clear();

        edge1 = new Edge(0,1);
        edge2 = new Edge(2,3);
        tmp.add(edge1);
        tmp.add(edge2);
        edges.put(1,(ArrayList<Edge>) tmp.clone());
        tmp.clear();

        edge1 = new Edge(1,3);
        edge2 = new Edge(5,7);
        tmp.add(edge1);
        tmp.add(edge2);
        edges.put(2,(ArrayList<Edge>) tmp.clone());
        tmp.clear();

        edge1 = new Edge(0,4);
        tmp.add(edge1);
        edges.put(3,(ArrayList<Edge>) tmp.clone());
        tmp.clear();

        edge1 = new Edge(5,2);
        tmp.add(edge1);
        edges.put(4,(ArrayList<Edge>) tmp.clone());
        tmp.clear();

        edge1 = new Edge(4,2);
        edge2 = new Edge(2,7);
        tmp.add(edge1);
        tmp.add(edge2);
        edges.put(5,(ArrayList<Edge>) tmp.clone());
        tmp.clear();

        Graph graph = new Graph(2, 3, edges);

        Path path = graph.dijkstra(0);
        assertArrayEquals(path.cost, new double[]{0.0, 1.0, 4.0, 4.0, 13.0, 11.0});
        assertArrayEquals(path.last, new int [] {-1,0,1,0,5,2});
    }
\end{verbatim}


\section{Błędy i wnioski}\label{header-n233}

Podczas przeprowadzania testów w czasie pisania programów zauważyłyśmy, że przedstawienie grafu w postaci macierzy sąsiedztwa nie jest optymalne: za bardzo obciąża projekt. Dlatego postanowiłyśmy zmienić postać przechowywania grafu. Graf jest przedstawiony w postaci listy z kluczami HashMap, gdzie kluczem jest numer wierzchołka, elementami - lista krawędzi, prowadzonych od niego. Krawędź jest przedstawiona klasą Edge, zawierającą numer wierzchołka końcowego i wagę. 


\begin{figure}[H]
\begin{center}
  \includegraphics[scale=0.5]{diagram(1).png}
  \end{center}
  \caption{Aktualny diagram klas}
  \label{fig:graf}
\end{figure}

\end{document}