\documentclass[]{article}
\usepackage{lmodern}
\usepackage{graphicx}
\usepackage{amssymb,amsmath}
\usepackage{ifxetex,ifluatex}
\usepackage{fixltx2e} % provides \textsubscript
\ifnum 0\ifxetex 1\fi\ifluatex 1\fi=0 % if pdftex
  \usepackage[T1]{fontenc}
  \usepackage[utf8]{inputenc}
\else % if luatex or xelatex
  \ifxetex
    \usepackage{mathspec}
  \else
    \usepackage{fontspec}
  \fi
  \defaultfontfeatures{Ligatures=TeX,Scale=MatchLowercase}
\fi
\renewcommand{\figurename}{Rys.}
% use upquote if available, for straight quotes in verbatim environments
\IfFileExists{upquote.sty}{\usepackage{upquote}}{}
% use microtype if available
\IfFileExists{microtype.sty}{%
\usepackage[]{microtype}
\UseMicrotypeSet[protrusion]{basicmath} % disable protrusion for tt fonts
}{}
\PassOptionsToPackage{hyphens}{url} % url is loaded by hyperref
\usepackage[unicode=true]{hyperref}
\hypersetup{
            pdftitle={Generacja grafu i odnalezienie najkrótszej ścieżki pomiędzy węzłami: graph},
            pdfauthor={Ulyana Petrashevich, Inga Maziarz}
            pdfborder={0 0 0},
            breaklinks=true}
\urlstyle{same}  % don't use monospace font for urls
\IfFileExists{parskip.sty}{%
\usepackage{parskip}
}{% else
\setlength{\parindent}{0pt}
\setlength{\parskip}{6pt plus 2pt minus 1pt}
}
\setlength{\emergencystretch}{3em}  % prevent overfull lines
\providecommand{\tightlist}{%
  \setlength{\itemsep}{0pt}\setlength{\parskip}{0pt}}
\setcounter{secnumdepth}{0}
% Redefines (sub)paragraphs to behave more like sections
\ifx\paragraph\undefined\else
\let\oldparagraph\paragraph
\renewcommand{\paragraph}[1]{\oldparagraph{#1}\mbox{}}
\fi
\ifx\subparagraph\undefined\else
\let\oldsubparagraph\subparagraph
\renewcommand{\subparagraph}[1]{\oldsubparagraph{#1}\mbox{}}
\fi
\renewcommand{\contentsname}{Spis treści}
% set default figure placement to htbp
\makeatletter
\def\fps@figure{htbp}
\makeatother


\title{\texttt{Specyfikacja implementacyjna}\\Generacja grafu i odnalezienie najkrótszej ścieżki pomiędzy węzłami - \texttt{graph} - Java}
\author{Ulyana Petrashevich, Inga Maziarz}
\date{17.05.2022}

\usepackage{fancyhdr}
\usepackage{lastpage}
\pagestyle{fancy}
\fancypagestyle{plain}{}
\fancyhf{}
\rhead{Specyfikacja implementacyjna}
\lhead{Graph - Java}
\cfoot{Strona \thepage \hspace{1pt} z \pageref{LastPage}}

\begin{document}
\maketitle
\tableofcontents
\thispagestyle{empty}
\newpage
\section{Informacje ogólne}\label{header-n231}
Program \texttt{graph} został napisany w języku Java w środowisku programistycznym ????. Współpraca i wersjonowanie odbywa się w repozytorium zdalnym w serwisie GitHub.

Wielkość okienka: 800 x 600.

\section{Opis modułów/pakietów}\label{header-n233}
Podrozdział: Moduł/pakiet - opis: W każdym z podrozdziałów opisujemy oddzielny moduł programu tzn. główne zadanie i powiązania między innymi modułami.

\section{Opis klas}\label{header-n233}
Podrozdział: Klasa - opis: W tym rozdziale opisujemy w jakim pakiecie znajduje się klasa, po czym dziedziczy, jakie klasy implementuje, jakie spełnia zadanie biznesowe, krótko opisujemy konstruktor.

Podpodrozdział: Metoda: Opisujemy oraz metody, a także pola wraz z nazwami i zwięzłą definicją. Prototyp: nagłówek metody. Zadanie: cóż czyni. Algorytm: jak to czyni - krok po kroku. Wartość zwracana: co zwraca.

\section{Słuchaczy akcji dla każdego z pól, przycisków, paneli}\label{header-n279}
 Jak widać w opisie logiki programu przechodzimy od ogółu do szczegółu i jeśli programujemy np. w Javie, i jeśli implementujemy GUI, to opisujemy też (wtedy będzie to rozdział 4) słuchaczy akcji dla każdego z pól, przycisków, paneli itp. Jest to w tym przypadku i w tej specyfikacji bardzo ważne, ponieważ dobry opis pola i jego słuchacza akcji odciąży w momencie programowania nas z zastanawiania się "jak to powinno być zrobione?", by móc spokojnie zająć się tym, by to po prostu było zrobione (jednakże należy pamiętać, że licho nigdy nie śpi i powinniśmy przeznaczyć niewielką część uwagi na poprawność rozwiązania).

\begin{itemize}
    \item Pole "Size"
    
    Pole będzie pobierało od użytkownika rozmiar grafu za pomocą getText().
    \item Pole "Edge weight range"
    
    Pole będzie pobierało od użytkownika zakres wartości wag krawędzi za pomocą getText().
    \item Przycisk opcji "Connectivity"
    
    Element Radio Box będzie pobierało wybraną opcję spójności grafu(jedną na raz) za pomocą ButtonGroup.getSelection().getActionCommand().
    \item Przycisk "Generate"
    
    Przycisk będzie uruchamiał wczytywanie wartości z pól "Size", "Edge weight range" i przycisku opcji "Connectivity", i uruchamiał generator grafu. Słuchaczem akcji występuje okno "Kratka".
    \item Przycisk "Read"
    
    Przycisk uruchomi wyświetlanie okienka do wyboru pliku do wczytania grafu, uruchomi jego wczytanie do programu i narysowanie w interfejsie graficznym.
    \item Przycisk "Save graph"
    
    Przycisk wyświetli okienko do wyboru/utworzenia pliku do zapisania obecnego grafu i uruchomi jego zapisywanie.
    \item Przycisk "Save path"
    
    Przycisk wyświetli okienko do wyboru/utworzenia pliku do zapisania wybranych ścieżek i uruchomi zapisywanie.
    \item Przycisk "Clear"
    
    Przycisk usunie wszystkie wyznaczone ścieżki w grafie.
    \item Przycisk "Delete"
    
    Przycisk usunie rysunek grafu z interfejsu.
    \item Modify color range
    
    Przycisk ?????
    \item Klikanie lewym klawiszem myszki na wierzchołek
    
    Klikanie uruchomi algorytm BFS do sprawdzenia spójności grafu, dalej uruchomi algorytm Dijkstry, przyjmując odznaczony wierzchołek jako początkowy i nada wierzchołkom na rysunku odpowiednie kolory. 
    
    \item Klikanie prawym klawiszem myszki na wierzchołek
    
    Klikanie uruchomi narysowanie ścieżki od poprzednio wybranego wierzchołka początkowego do zaznaczonego.
\end{itemize}

\section{Budowa GUI}\label{header-n279}

\begin{figure}[h!]
\begin{center}
  \includegraphics[scale=0.6]{graf.png}
  \end{center}
  \caption{Projekt głównego okna interfejsu}
  \label{fig:graf}
\end{figure}

\section{Testowanie}\label{header-n281}

\section{Diagram klas}\label{header-n283}
\end{document}